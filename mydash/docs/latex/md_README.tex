
\begin{DoxyItemize}
\item Author\-: Noah Rudin
\item Class\-: C\-S453 \mbox{[}Operating Systems\mbox{]} Section 2
\end{DoxyItemize}

\subsection*{Overview}

The mydash program is a shell that accepts arguments next to commands and other, more basic commands such as cd, ls, pwd. and exit. In addition, the user may append an ampersand after any input to send the desired process into the background, where the shell will report its completion after hitting enter. The user may check on the status of background processes with the command \char`\"{}jobs\char`\"{}.

\subsection*{Manifest}

\subsubsection*{Makefile -\/ supplies rules to compile the C code into executable}

\subsubsection*{mydash.\-c -\/ main program code that contains most mydash functionality}

\subsubsection*{mydash.\-h -\/ header file for declaring rules and functions for mydash.\-c}

\subsubsection*{R\-E\-A\-D\-M\-E.\-md -\/ this file, provides a user with information regarding mydash}

\subsubsection*{valgrind.\-supp -\/ a list of rules to suppress on Valgrind}

\subsubsection*{Test\-Cases -\/ tests written to help test mydash, containg commands with fringe cases and odd processes}

\subsection*{Building the project}

To build this code, first install readline from\-:

\href{https://tiswww.case.edu/php/chet/readline/rltop.html}{\tt https\-://tiswww.\-case.\-edu/php/chet/readline/rltop.\-html}

Next, compile the source code by typing \char`\"{}make\char`\"{} within the p1 folder.

To run with Valgrind, run \char`\"{}make valgrind\char`\"{} in the p1 folder.

\subsection*{Features and usage}

To use mydash, type ./mydash to launch the executable. While the program runs, it should function similar to a command-\/line terminal. Mydash contains basic command functionality such as cd, ls, pwd, and exit. Additionally, the user may type an ampersand after a command (e.\-g.-\/ \char`\"{}ls\&\char`\"{}) to start the argument process in the background. Typing in \char`\"{}jobs\char`\"{} will output the status of all background processes. Each completed job will be reported only once.

To use mydash with valgrind, type valgrind --leak-\/check=full ./mydash to launch mydash with valgrind feedback.

\subsection*{Testing}

To test the mydash program, I leaned on the supplied unit tests, as well as generating my own that input several odd arguments to try to break the program, such as putting unique processes in the background. These tests allowed me to realize different areas where common commands would work, but odd cases that should still technically work would break. For example, typing in \char`\"{}sleep 5\&\char`\"{} twice and then hitting jobs originally would output the process four times instead of two.

\subsubsection*{Valgrind}

After extensive testing and patching with Valgrind, the leak situation is much improved, but the bill of health is still not yet perfect. The exit command does not perfectly free all lists or jobs, but I am happy that most, if not all, of the other possible commands do not report any leaks. Exit is the only known problematic command.

\subsubsection*{Known Bugs}

None, outside of the Valgrind error within \char`\"{}exit\char`\"{} command

\subsection*{Reflection and Self Assessment}

I spent the biggest amount of time during this project working on the print\-List method that was supplied with the sample code. I was frustrated initially about how I could not see how the print\-List function worked, but I ended up creating my own print\-All\-Jobs method that traversed the jobs linked list and output the status of each unreported/running job. In addition to this, I felt that handling signals required an unexpected number of hours to complete. Specfically, I needed to teach myself the distinctions between several different types of signals and signal codes. I believe the current version of mydash contains the correct signal handling methods and flags, but it is definitely a topic I would like to learn more about within the field of C programming.

While that part of the project took the longest time-\/wise, the most trying part of the project patience-\/wise was resolving the many valgrind leaks within mydash. I improved a few things from mydash part 1, in which I failed to trim the null-\/terminating character at the end of a string. At the end of the day, I would have liked to go back and resolve all of my errors within Valgrind to create safe code.

\subsection*{Sources used}

\href{https://stackoverflow.com/questions/16828378/readline-get-a-new-prompt-on-sigint}{\tt Learning About Long\-Jmp}

\href{https://github.com/adam-p/markdown-here/wiki/Markdown-Cheatsheet#links}{\tt Link Help} 